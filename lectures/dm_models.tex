%lualatex --shell-escape  <name>.tex; 
%view <name>.pdf
%\documentclass[smaller, proffesionalfonts]{beamer}
\documentclass[proffesionalfonts]{beamer}
\usecolortheme[named=gray]{structure}
\mode<presentation> {
%\usetheme{Madrid} % My favorite!
%\usetheme{Boadilla} % Tretty neat, soft color.
%\usetheme{default}
\usetheme{Warsaw}
%\usetheme{Bergen} % This template has nagivation on the left
%\usetheme{Frankfurt} % Similar to the default with an extra region at the top.
%\usecolortheme{seahorse} % Simple and clean template
%\usetheme{Darmstadt} % not so good
% Uncomment the following line if you want page numbers and using Warsaw theme
% \setbeamertemplate{footline}[page number]
\setbeamercovered{transparent}
%\setbeamercovered{invisible}
% To remove the navigation symbols from the bottom of slides%
\setbeamertemplate{navigation symbols}{} 
}

\usepackage{ifpdf}
\usepackage{ifluatex}
\ifluatex
\usepackage{polyglossia}   
\setmainlanguage{polish}
\usepackage{amsmath}
\usepackage{mathtools}
\def\bm{\boldsymbol}
\usepackage{lualatex-math}
\else
\usepackage{graphicx}
\usepackage{polski} 
\usepackage[utf8x]{inputenc}
\usepackage{bm} 
\fi
\usepackage{ucs}
\usepackage{listings}
\usepackage{hyperref}
\usepackage{ragged2e}
%\apptocmd{\frame}{}{\justifying}{} % Allow optional arguments after frame.
%\apptocmd{\block}{}{\justifying}{} % Allow optional arguments after frame.
% For typesetting bold math (not \mathbold)

\usepackage{pgf,tikz}
\usetikzlibrary{graphdrawing}
\usetikzlibrary{graphs}
\usegdlibrary{trees}
\usetikzlibrary{intersections}
\usetikzlibrary{arrows,shapes,automata,positioning,calc}
\usepackage{pgfplots}
\usepackage{pgfplotstable}
%\usepackage[active,tightpage,pdftex]{preview}
%\PreviewEnvironment{tikzpicture}
\newcommand{\mycircle}[1]{\tikz{\filldraw[draw=#1,fill=#1] (0,0) circle [radius=0.3em];}}
\DeclareMathOperator{\sgn}{sign}
\usepackage{tkz-euclide}

%\logo{\includegraphics[height=0.6cm]{yourlogo.eps}}
%
\title[EARIN]{Data Mining Lectures - Models, Model Ensembles}
%
\author{Piotr Wąsiewicz}
\institute[ICS PW]
{
Institute of Computer Science\\
\medskip
{\emph{pwasiewi@elka.pw.edu.pl}}
}
\date{\today}
% \today will show current date. 
% Alternatively, you can specify a date.

\begin{document}
%\lstset{language=SQL}

\lstset{
    language=R,
    inputencoding=utf8x,
    extendedchars=\true,
    literate={ą}{{\k{a}}}1
             {Ą}{{\k{A}}}1
             {ę}{{\k{e}}}1
             {Ę}{{\k{E}}}1
             {ó}{{\'o}}1
             {Ó}{{\'O}}1
             {ś}{{\'s}}1
             {Ś}{{\'S}}1
             {ł}{{\l{}}}1
             {Ł}{{\L{}}}1
             {ż}{{\.z}}1
             {Ż}{{\.Z}}1
             {ź}{{\'z}}1
             {Ź}{{\'Z}}1
             {ć}{{\'c}}1
             {Ć}{{\'C}}1
             {ń}{{\'n}}1
             {Ń}{{\'N}}1
}
%\lstset{language=c}
%\lstset{backgroundcolor=\color{brown}}
%\lstset{linewidth=90mm}
%\lstset{frameround=tttt}
\lstset{frameround=trbl}
\lstset{keywordstyle=\color{black}\bfseries}
\lstset{commentstyle=\textit, stringstyle=\upshape,showspaces=false}
%\lstset{commentstyle=\textit}

\begin{frame}
\titlepage
\end{frame}

\begin{frame}
\frametitle{Literature}
\begin{block}{Books}
\begin{itemize}
\item P. Cichosz, Data Mining algorithms: explained using R, Wiley 2015
\end{itemize}
\end{block}
\begin{block}{Internet sites}
\begin{itemize}
\item https://github.com/pwasiewi/earin
\item https://github.com/pwasiewi/dokerz/tree/master/rstudio
\item http://www.wiley.com/go/data\_mining\_algorithms
\item http://www.stat.wisc.edu/$\tilde{\ }$larget/stat302/chap3.pdf
\item http://www.wekaleamstudios.co.uk/supplementary-material/
\end{itemize}
\end{block}
\end{frame}

\begin{frame}
\frametitle{Data Sets}
\begin{block}{Tables}
\begin{itemize}
\item A table (a set) consist of several columns, variables, input attributes or rows as vectors of attribute values.
\item A limited dataset contains also a depended on input attributes variable called the target variable, the target attribute, the class label, the class concept, classes, the target concept, the group identifier, the response variable (regression).
\item Each row can be called an instance, an object descibed by attribute values.
\item Attributes can be nominal with a finite number of discrete values, ordinal - like nominal with a total order relation, continouos (numerical, linear) having numerical values.
\item The target attribute classifies instances (objects) into classes or clusters them into groups.
\end{itemize}
\end{block}
\scriptsize
\begin{block}{}
\begin{itemize}
\item Baskets consists e.g. of different length sets of product ids (ordinal numbers).
\end{itemize}
\end{block}
\end{frame}


\begin{frame}
\footnotesize 
\frametitle{Statistical elements}
\begin{block}{}
\begin{itemize}
\item If the p-value is below the significance level, then the null hypothesis is rejected (no relationship in the given domain).
\item False positive (type I error) rejected null hypothesis is actually true. This risk increases with a larger significance level.
\item False negative (type II error) rejected alternative hypothesis is actually true. This risk increases with a smaller significance level.
\item Pearson's linear correlation coefficient measures the strength of linear relationship between two continuous attributes, null hypothesis - 0 value.
\item Spearman's rank correlation also has values from -1 to 1 for increasing relationship.
\item For discrete attributes Chi2 and the loglikelihood ratio (G-test) tests can be taken.
\item For mixed attributes t-test (with Mann-Whitney-Wilcoxon nonparametric alternative) and f-test - one-way ANOVA (with Kruskal-Wallis alternative) can be used.
\item The population is to the sample as the sample is to the bootstrap samples (samples with replacement).
\end{itemize}
\end{block}
\end{frame}

\begin{frame}
\frametitle{The model e.g. the decision tree}
\begin{block}{Create the model to return a prediction after taking an instance as an argument}
\begin{itemize}
\item Made a classifier which is trained on a training set with a target attribute (supervised learning tasks using some expert knowledge put in the target variable) and predicts class label values for rows without this target classes. 
\item Find a regression target values based on input data.
\item Without any expert knowledge cluster groups using the similarity structure discovered in the input data and write groups identifiers to the target attribute.
\end{itemize}
\end{block}
\end{frame}

\begin{frame}
\frametitle{Model performance}
\begin{block}{Quality of predictions}
\begin{itemize}
\item The expected quality of predictions on the whole domain including previously unseen instances
\end{itemize}
\end{block}
\begin{block}{Overfitting}
\begin{itemize}
\item A model perfoms worse on the whole domain than on the training dataset, where e.g. it has excellent performance.
\end{itemize}
\end{block}
\end{frame}

\begin{frame}
%\frametitle{}
\begin{block}{Model ensembles}
\begin{itemize}
\item For three binary classification models with independent mistakes voting reduces error if base model error below 1/2 ($\epsilon^3+3\epsilon^2(1-\epsilon)<\epsilon$)
\end{itemize}
\end{block}
\begin{block}{Base model generation}
\begin{itemize}
\item Different sets of training instances: use a different set of training instances to create each base model.
\item Different sets of attributes: use a different set of attributes to create each base model.
\item Different algorithms: use a different algorithm to create each base model.
\item Different parameter setups: use a different algorithm parameter setup to create each base model.
\item Algorithm randomization: use independent runs of a non-deterministic algorithm to create each base model.
\end{itemize}
\end{block}
\end{frame}

\begin{frame}
\frametitle{Ensemble prediction}
\begin{block}{}
\begin{itemize}
\item Voting: combine base model predictions by (possibly weighted) voting.
\item Probability averaging: combine base model class probability predictions by (possibly weighted) averaging.
\item Using as attributes: create a combined model using base model predictions as attributes.
\end{itemize}
\end{block}
\end{frame}

\begin{frame}
%\frametitle{Bagging}
\begin{block}{Bagging}
\begin{itemize}
\item Base models created using a single algorithm applied to multiple bootstrap samples of the training set samples drawn at random with replacement - 63.8\% bag, 37.2\% out-of-bag (OOB).
\item Ensemble prediction by (unweighted) voting.
\item Requires an unstable algorithm for sufficient base model diversity e.g., decision trees, but after some number of base models it is stabilized.
\item Moderate prediction improvement, but a useful stabilization effect.
\end{itemize}
\end{block}
\begin{block}{Stacking}
\begin{itemize}
\item Different base model algorithms.
\item Very difficult to design.
\end{itemize}
\end{block}
\end{frame}

\begin{frame}
\frametitle{Boosting}
\begin{block}{}
\begin{itemize}
\item Increase base model diversity by shifting focus during model creation.
\item Force subsequent base model to compensate deficiencies of the preceding ones.
\item Typically used with weight-sensitive classification algorithms increase weights of instances incorrectly classified by the previous model.
\item Ensemble prediction by weighted voting.
\item Substantial prediction improvement.
\item Simple base models work well e.g., small decision trees or decision stumps.
\item Can not be executed in parallel.
\item AdaBoost - weighted base models and attributes.
\end{itemize}
\end{block}
\end{frame}

\begin{frame}
\small
\frametitle{Random forest}
\begin{block}{}
\begin{itemize}
\item Bagging combined with algorithm randomization for increased base model diversity.
\item Randomized decision trees sample attributes at each node prior to split selection.
\item Ensemble prediction by voting and substantial prediction improvement.
\item Maximum or near-maximum fit of base models individual overfitting gets compensated by combining many randomized trees.
\item Estimate prediction quality using OOB (Out-of-bag) instances.
\item For each training instance: generate and combine predictions of those and only those trees for which the instance is OOB
\item Use these predictions to calculate the misclassification error (or another prediction quality indicator).
\item Estimate the predictive utility of attributes by measuring error on the mutated data.
\end{itemize}
\end{block}
\end{frame}

\begin{frame}
\small
\frametitle{Case study: data preparation}
\begin{block}{}
\begin{itemize}
\item Attributes not related to the class target have to be removed.
\item Attributes with too many missing values (NA) may be removed or go through imputation - filling missing values with means or medians for continuous attributes or with modes (most frequent values) for discrete ones.
\item Attributes with many outliers may be removed or another way is to get rid of these outliers or to use median.
\item For some algorithms attributes can be standardized $(x-mean)/sd$ or/and normalized $(x - min)/(max -min)$ .
\item Attributes strongly correlated should be removed except the best one e.g. with less missing values (NA) or with a fewer outliers.
\item Attributes with too many discrete values can aggregate them to max number of 32 (it is suitable for many random forest algorithms).
\item The given dataset is divided into three sets: training, validating and testing data. 
\end{itemize}
\end{block}
\end{frame}

\begin{frame}
\frametitle{Case study}
\begin{block}{Training phase}
\begin{itemize}
\item Training of your model based on the training dataset.
\end{itemize}
\end{block}
\begin{block}{Validation phase}
\begin{itemize}
\item Estimation how well your model is trained and how to find model best properties, training algorithm parameters.
\end{itemize}
\end{block}
\begin{block}{Testing phase}
\begin{itemize}
\item At the end of the process checking quality of the trained and validated model using the testing dataset.
\end{itemize}
\end{block}
\end{frame}
% End of slides
\end{document} 

\begin{frame}
\frametitle{}
\begin{block}{}
\begin{itemize}
\item 
\end{itemize}
\end{block}
\end{frame}

\begin{frame}[fragile]
\scriptsize
\frametitle{}
\begin{lstlisting}
\end{lstlisting}
\end{frame}

\begin{frame}
%\frametitle{}
%\includegraphics[width=11.8cm]{11gdiag\_01.jpg}
\small
\begin{block}{}
\begin{itemize}
\item 
\end{itemize}
\end{block}
\end{frame}

