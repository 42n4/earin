\RequirePackage{atbegshi} 
\documentclass{beamer}
\usecolortheme[named=gray]{structure}
\mode<presentation> {
%\usetheme{Madrid} % My favorite!
%\usetheme{Boadilla} % Pretty neat, soft color.
%\usetheme{default}
\usetheme{Warsaw}
%\usetheme{Bergen} % This template has nagivation on the left
%\usetheme{Frankfurt} % Similar to the default with an extra region at the top.
%\usecolortheme{seahorse} % Simple and clean template
%\usetheme{Darmstadt} % not so good
% Uncomment the following line if you want page numbers and using Warsaw theme
% \setbeamertemplate{footline}[page number]
\setbeamercovered{transparent}
%\setbeamercovered{invisible}
% To remove the navigation symbols from the bottom of slides%
\setbeamertemplate{navigation symbols}{} 
}

\usepackage{ifpdf}
%\usepackage{graphicx}
\usepackage{polski} 
\usepackage[utf8x]{inputenc}
\usepackage{ucs}
\usepackage{listings}
\usepackage{hyperref}

%\usepackage{bm} 
% For typesetting bold math (not \mathbold)
%\logo{\includegraphics[height=0.6cm]{yourlogo.eps}}
%
\title[EARIN]{Data Mining Lectures - Case Study}
%
\author{Piotr Wąsiewicz}
\institute[ICS PW]
{
Institute of Computer Science\\
\medskip
{\emph{pwasiewi@elka.pw.edu.pl}}
}
\date{\today}
% \today will show current date. 
% Alternatively, you can specify a date.

\begin{document}
%\lstset{language=SQL}

\lstset{
    language=R,
    inputencoding=utf8x,
    extendedchars=\true,
    literate={ą}{{\k{a}}}1
             {Ą}{{\k{A}}}1
             {ę}{{\k{e}}}1
             {Ę}{{\k{E}}}1
             {ó}{{\'o}}1
             {Ó}{{\'O}}1
             {ś}{{\'s}}1
             {Ś}{{\'S}}1
             {ł}{{\l{}}}1
             {Ł}{{\L{}}}1
             {ż}{{\.z}}1
             {Ż}{{\.Z}}1
             {ź}{{\'z}}1
             {Ź}{{\'Z}}1
             {ć}{{\'c}}1
             {Ć}{{\'C}}1
             {ń}{{\'n}}1
             {Ń}{{\'N}}1
}
%\lstset{language=c}
%\lstset{backgroundcolor=\color{brown}}
%\lstset{linewidth=90mm}
%\lstset{frameround=tttt}
\lstset{frameround=trbl}
\lstset{keywordstyle=\color{black}\bfseries}
\lstset{commentstyle=\textit, stringstyle=\upshape,showspaces=false}
%\lstset{commentstyle=\textit}

\begin{frame}
\titlepage
\end{frame}

\begin{frame}
\small
\frametitle{Case study}
\begin{block}{Data preparation}
\begin{itemize}
\item Attributes not related to the class target have to be removed.
\item Attributes with too many missing values (NA) may be removed or go through imputation - filling missing values with means or medians for continuous attributes or with modes (most frequent values) for discrete ones.
\item Attributes with many outliers may be removed or another way is to get rid of these outliers or to use median.
\item For some algorithms attributes can be standardized $(x-mean)/sd$ or/and normalized $(x - min)/(max -min)$ .
\item Attributes strongly correlated should be removed except the best one e.g. with less missing values (NA) or with a fewer outliers.
\item Attributes with too many discrete values can aggregate them to max number of 32 (it is suitable for many random forest algorithms).
\item The given dataset is divided into three sets: training, validating and testing data. 
\end{itemize}
\end{block}
\end{frame}

\begin{frame}
\frametitle{Case study}
\begin{block}{Training phase}
\begin{itemize}
\item Training of your model based on the training dataset.
\end{itemize}
\end{block}
\begin{block}{Validation phase}
\begin{itemize}
\item Estimation how well your model is trained and how to find model best properties, training algorithm parameters.
\end{itemize}
\end{block}
\begin{block}{Testing phase}
\begin{itemize}
\item At the end of the process checking quality of the trained and validated model using the testing dataset.
\end{itemize}
\end{block}
\end{frame}

% End of slides
\end{document} 

\begin{frame}
\frametitle{}
\begin{block}{}
\begin{itemize}
\item 
\end{itemize}
\end{block}
\end{frame}

\begin{frame}[fragile]
\scriptsize
\frametitle{}
\begin{lstlisting}
\end{lstlisting}
\end{frame}

\begin{frame}
%\frametitle{}
%\includegraphics[width=11.8cm]{11gdiag\_01.jpg}
\small
\begin{block}{}
\begin{itemize}
\item 
\end{itemize}
\end{block}
\end{frame}

