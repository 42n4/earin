%lualatex --shell-escape  <name>.tex; 
%view <name>.pdf
%\documentclass[smaller, proffesionalfonts]{beamer}
\documentclass[proffesionalfonts]{beamer}
\usecolortheme[named=gray]{structure}
\mode<presentation> {
%\usetheme{Madrid} % My favorite!
%\usetheme{Boadilla} % Tretty neat, soft color.
%\usetheme{default}
\usetheme{Warsaw}
%\usetheme{Bergen} % This template has nagivation on the left
%\usetheme{Frankfurt} % Similar to the default with an extra region at the top.
%\usecolortheme{seahorse} % Simple and clean template
%\usetheme{Darmstadt} % not so good
% Uncomment the following line if you want page numbers and using Warsaw theme
% \setbeamertemplate{footline}[page number]
\setbeamercovered{transparent}
%\setbeamercovered{invisible}
% To remove the navigation symbols from the bottom of frames%
\setbeamertemplate{navigation symbols}{} 
}

\usepackage{ifpdf}
\usepackage{ifluatex}
\ifluatex
\usepackage{polyglossia}   
\setmainlanguage{polish}
\usepackage{amsmath}
\def\bm{\boldsymbol}
\usepackage{lualatex-math}
\else
\usepackage{graphicx}
\usepackage{polski} 
\usepackage[utf8x]{inputenc}
\usepackage{bm} 
\fi
\usepackage{ucs}
\usepackage{listings}
\usepackage{hyperref}
\usepackage{ragged2e}
%\apptocmd{\frame}{}{\justifying}{} % Allow optional arguments after frame.
%\apptocmd{\block}{}{\justifying}{} % Allow optional arguments after frame.
% For typesetting bold math (not \mathbold)

\usepackage{pgf,tikz}
\usetikzlibrary{graphdrawing}
\usetikzlibrary{graphs}
\usegdlibrary{trees}
\usetikzlibrary{intersections}
\usetikzlibrary{arrows,shapes,automata,positioning,calc}
\usepackage{pgfplots}
\usepackage{pgfplotstable}
%\usepackage[active,tightpage,pdftex]{preview}
%\PreviewEnvironment{tikzpicture}
\DeclareMathOperator{\sgn}{sign}
\usepackage{tkz-euclide}

\newcommand{\mycircle}[1]{\tikz{\filldraw[draw=#1,fill=#1] (0,0) circle [radius=0.3em];}}
\newcommand{\s}[1]{{\scriptscriptstyle 
    \fontfamily{cmss}
    \selectfont
#1}}
\newcommand{\un}[1]{\underline{#1}}
\newcommand{\sz}[1]{\scriptsize{#1}}
\newcommand{\arial}[1]{{ #1}}
\newcommand{\faup}{ }
\newcommand{\arit}[1]{{\it #1 \/}}
\newcommand{\arbf}[1]{\bf{\arial{#1}\/}\faup}
\newcommand{\SA}[1]{$#1$\faup}
\newcommand{\argmin}{\arg\!\min} 
\newcommand{\argmax}{\arg\!\max} 
%\newcommand{\Treebox}[1]{% \Tr{\psframebox{#1}}}

%\logo{\includegraphics[height=0.6cm]{yourlogo.eps}}
%
\title[EARIN]{Data Mining Lectures - Decision trees}
%
\author{Piotr Wasiewicz}
\institute[ICS PW]
{
Institute of Computer Science\\
\medskip
{\emph{pwasiewi@elka.pw.edu.pl}}
}
\date{\today}
% \today will show current date. 
% Alternatively, you can specify a date.

\begin{document}

\begin{frame}
\frametitle{Probability}
\begin{center}
\large
\begin{description}
\item[$P(A)$]$= \frac{T^A}{T}$
\item[$P(A)$] - the measure of likelihood that an event $A$ will occur
\item[$T^A$] - all possible results associated with the event $A$
\item[$T$] - all possible results 
\end{description}
\normalsize
\end{center}
\end{frame}

\begin{frame}\frametitle{Conditional probability}
\large
\[P(C|A)=\frac{P(C\cap A)}{P(A)}\textsf{\faup\ \ - conditional probability that a patient }\] 
\ \ \ \ \ has a disease $C$, if he has symptoms $A$
\[P(A|C)=\frac{P(A\cap C)}{P(C)}\textsf{\faup\ \  - conditional probability that a patient }\]
\ \ \ \ \ has symptoms $A$, if he has a disease $C$
\begin{description}
\item[$P(C\cap A)$] - probability that a patient has a disease $C$ and symptoms $A$
\item[$P(C)$] - probability that a patient has a disease $C$
\item[$P(A)$] - probability of symptoms
\end{description}
\normalsize
\end{frame}

\begin{frame}\frametitle{Bayes theorem}
\begin{center}
\large
\[P(C|A)=\frac{P(C\cap A)}{P(A)}\]  
\[P(A|C)=\frac{P(A\cap C)}{P(C)}\]\\
\Large
\[P(C|A)=\frac{P(A|C)*P(C)}{P(A)}\]
\normalsize
\end{center}
\end{frame}

\begin{frame}\frametitle{Conditional probability table}
\begin{center}
\small
Table describing conditional probabilities of diseases, \\
where the given symptom was observed:\\ 
\ \\
   \begin{tabular}{||r||c|c|c|c||}
   \hline \hline
    & influenza $C_1$ & cold $C_2$ & pneumonia $C_3$ & allergy  $C_4$ \\
   \hline
   headache $A_1$ & $P(C_1|A_1)$ & $P(C_2|A_1)$ & $P(C_3|A_1)$ & $P(C_4|A_1)$  \\
   \hline
   cough $A_2$ & $P(C_1|A_2)$ & $P(C_2|A_2)$ & $P(C_3|A_2)$ & $P(C_4|A_2)$  \\
   \hline
   sneeze $A_3$  & $P(C_1|A_3)$ & $P(C_2|A_3)$ & $P(C_3|A_3)$ & $P(C_4|A_3)$  \\
   \hline
   temperature $A_4$ & $P(C_1|A_4)$ & $P(C_2|A_4)$ & $P(C_3|A_4)$ & $P(C_4|A_4)$  \\
   \hline \hline
   \end{tabular}
\[\sum_{i=1}^{n}P(A_i)=1\qquad \sum_{j=1}^{m}P(C_j|A_i)=1\qquad P(C_j)=\sum_{i=1}^{n}P(A_i)*P(C_j|A_i)\]
\[P(A_i|C_j)=\frac{P(A_i)*P(C_j|A_i)}{P(C_j)}\qquad P(C_j|A_i)=\frac{P(C_j)*P(A_i|C_j)}{P(A_i)}\]
\normalsize
\end{center}
\end{frame}

\begin{frame}\frametitle{More general Bayes Theorem formula}
Bayes theorem has the more general form for \un{many} diseases and \un{many} symptoms:
\begin{center}
\large
\[P(C_j|A_{i1}\cap\ldots\cap A_{ik})=\frac{P(C_j)*P(A_{i1}|C_j)*\ldots *P(A_{ik}|C_j)}{\displaystyle\sum_{l=1}^{n}P(C_l)*P(A_{i1}|C_l)*\ldots *P(A_{ik}|C_l)}\]
\normalsize
\end{center}
\end{frame}

\begin{frame}\frametitle{Bayes Theorem: the comparison of equivalent sets and events}
\begin{center}
\normalsize
   \begin{tabular}{p{5.5cm}|p{5.5cm}}
   $\Omega$ - a space of independent elementary observed results; $A\in 2^\Omega\Rightarrow A'\in 2^\Omega$ - complementarity; $A,B\in 2^\Omega\Rightarrow A\cup B\in 2^\Omega$ - additivity  &  $F$ - the independent rule set such that $a\in F\Leftrightarrow b\notin F-\{\mathbf{0},a\}$ this means $b\land\neg a=\mathbf{0}$ \\
   \hline
   $(2^\Omega,\cup,\cap,',\Omega,\phi)$ & $(F,\lor,\land,\neg,\mathbf{1},\mathbf{0})$ \\
   \hline
   $P(\phi)=0\qquad P(\Omega)=1$ & $P(\mathbf{0})=0\qquad P(\mathbf{1})=1$ \\
   \hline
   $A\cap A'=\phi\quad A\cup A'=\Omega$ & $a\land\neg a=\mathbf{0}\quad a\lor\neg a=\mathbf{1}$ \\
   \hline
   $\forall A,B\in 2^\Omega\quad A\cap B=\phi$ & $\forall a,b\in F\quad a\land b=\mathbf{0}$ \\
   $P(A\cup B)=P(A)+P(B)$ & $P(a\lor b)=P(a)+P(b)$\\
   \hline
   $\forall A\in 2^\Omega\quad P(A)+P(A')=1$ & $\forall a\in F\quad P(a)+P(\neg a)=1$  \\
   \hline
   $A\subseteq B\qquad P(A)\le P(B)$ & $(a\Rightarrow b)=\mathbf{1}\qquad P(a)\le P(b)$ \\
   \end{tabular}
\normalsize
\end{center}
\end{frame}

\begin{frame}\frametitle{Bayes model}
\begin{center}
\large
Bayes rule\\
\[P(h|e)=\frac{P(e|h)P(h)}{P(e)}\]\\
where $h$ means hypothesis and $e$ denotes an event.\\
Such a rule is just an another form of the usual rule:\\
\ \\
\Large
$e\Rightarrow h$
\normalsize
\end{center}
\end{frame}

\begin{frame}\frametitle{Bayes Theorem}
\begin{center}
\normalsize
$\exists\ H=\{h_1,\ldots,h_n\}$, where $\forall i\ne j\quad h_i\land h_j=\mathbf{0}\quad\displaystyle\bigcup_{i=1}^{n}h_i=\mathbf{1},\quad P(h_i)>0,\quad i=1,\ldots,n$\\
$\exists\ \{e_1,\ldots,e_m\}$, where $\displaystyle P(e_1,\ldots,e_m|h_i)=\prod_{j=1}^m P(e_j|h_i),\ i=1,\ldots,n\Leftrightarrow$\\
$\Leftrightarrow\forall e_j,h_i\quad e_j$ conditionally independent on $h_i$\\[3mm]
\large
$P(h_i|e_1,\ldots,e_m)=\frac{\displaystyle P(e_1,\ldots,e_m|h_i)P(h_i)}{\displaystyle\sum_{k=1}^{n}P(e_1,\ldots,e_m|h_k)P(h_k)}$\\
$P(h_i|e_1,\ldots,e_m)=\frac{\displaystyle\prod_{j=1}^m P(e_j|h_i)}{\displaystyle\sum_{k=1}^n \prod_{j=1}^m P(e_j|h_k)P(h_k)}P(h_i)$
\normalsize
\end{center}
\end{frame}

\begin{frame}\frametitle{PROSPECTOR modifications (1976)}
\begin{center}
\large
An additional assumption: $\displaystyle P(e_1,\ldots,e_m|\neg h_i)=\prod_{j=1}^m P(e_j|\neg h_i),\ i=1,\ldots,n$\\
New Bayes rule: $\displaystyle P(\neg h|e)=\frac{P(e|\neg h)P(\neg h)}{P(e)}$ or\\[1mm]
$\displaystyle \frac{P(h|e)}{P(\neg h|e)}=\frac{P(e|h)}{P(e|\neg h)}\frac{P(h)}{P(\neg h)}$\\[1mm]
$\displaystyle O(h)=\frac{P(h)}{P(\neg h)}$ - a chance \un{a priori}\\[1mm]
$\displaystyle O(h|e)=\frac{P(h|e)}{P(\neg h|e)}$ - a chance \un{a posteriori}\\[1mm]
A reliability coefficient: $\displaystyle \lambda=\frac{P(e|h)}{P(e|\neg h)}\Rightarrow O(h|e)=\lambda O(h)$
\normalsize
\end{center}
\end{frame}

\begin{frame}\frametitle{Further PROSPECTOR modifications}
\begin{center}
\large
In a general case: $\displaystyle O(h_i|e_1,\ldots,e_m)=O(h_i)\prod_{k=1}^m\lambda_{k_i}$,\\
where $\displaystyle \lambda_{k_i}=\frac{P(e_k|h_i)}{P(e_k|\neg h_i)}$ \\[8mm]
$\displaystyle \overline{\lambda}=\frac{P(\neg e|h)}{P(\neg e|\neg h)}\Rightarrow O(h|\neg e)=\overline{\lambda}O(h)$\\[2mm]
\parbox[b]{115mm}{
Coefficients $\lambda$ i $\overline{\lambda}$ are defined a priori. $\lambda$ denotes observation sufficiency $e$ (especially for $\lambda\gg 1$) and $\overline{\lambda}$ denotes necessity $e$ (especially for $0\le\overline{\lambda}\le 1$).
}
\normalsize
\end{center}
\end{frame}

\begin{frame}\frametitle{Bayes model disadvantages}
\begin{center}
\large
\begin{itemize}
\item
Assumptions are not accomplished.
\item
Ignorance is hidden in a priori probabilities.
\item
Probabilities are known only for elementary observed independently events, but not for their sets.
\item
Probabilities are for both negative and positive events at the same time.
\end{itemize}
\normalsize
\end{center}
\end{frame}

\begin{frame}
\frametitle{Naive Bayes classifier assumptions}
\begin{itemize}
\item
Each instance $x$ described by attribute values $a(x)=\langle a_1(x), a_2(x) \ldots a_n(x) \rangle$, where $a_i(x)$ is the given value of the attribute $a_i$ ($a_i(x)\in \{a_{ij}\},\ j\in (1\ldots A_i$)). 
\item
Attribute values $a_i(x)$ of instances $x$ are conditionally independent given the target class $C_k$.
\item
It is so called Naive Bayes assumption: 
$\displaystyle P(a(x)|C_k) = \prod_i P(a_i(x) | C_k)$\\
which is usually not true, but incorrect class probabilities very often permit correct classification.
\item 
Conditional probabilities of attribute values $a_i(x)$ given the class $C_k$ are $P(a_i(x) | C_k) = P_{T^{C_k}}(a_i(x)) = \frac{|T^{C_k}_{a_i(x)}|}{|T^{C_k}|}$.
\item
$\displaystyle P(C_k|a(x)) = \frac{P(C_k) \prod_i P(a_i(x) | C_k)}{\sum_{C_l\in C}P(C_l) \prod_i P(a_i(x) | C_l)}$
\end{itemize}
\end{frame}

\begin{frame}
\frametitle{Naive Bayes classifier}
\begin{itemize}
\item
The final Naive Bayes classifier hypothesis $h(x)$ predicting the correct class is just the greatest conditional probability:
$\displaystyle P(C_k|a(x)) = \frac{P(C_k) P(a(x) | C_k)}{\sum_{C_l\in C}P(C_l) P(a(x) | C_l)}$
\item
$\displaystyle P(C_k|a(x)) = \frac{\displaystyle P(C_k) \prod_i P(a_i(x) | C_k)}{\displaystyle \sum_{C_l\in C}P(C_l) \prod_i P(a_i(x) | C_l)}$
\item
$\displaystyle h(x) = \argmax_{C_k \in C} P(C_k|a(x))$
\item
In a case of not present values in training instances to prevent prediction errors the number of values $A_i$ of the attribute $a_i$ is added to conditional probability:\\
$P(a_i(x) | C_k) = P_{T^{C_k}}(a_i(x)) = \frac{|T^{C_k}_{a_i(x)}|+1}{|T^{C_k}|+A_i}$.
$\displaystyle $
\end{itemize}
\end{frame}

\end{document}

\begin{frame}
\frametitle{}
\begin{block}{}
\begin{itemize}
\item
\end{itemize}
\end{block}
\end{frame}

