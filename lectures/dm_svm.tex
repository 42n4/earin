%lualatex --shell-escape  <name>.tex; 
%view <name>.pdf
%\documentclass[smaller, proffesionalfonts]{beamer}
\documentclass[proffesionalfonts]{beamer}
\usecolortheme[named=gray]{structure}
\mode<presentation> {
%\usetheme{Madrid} % My favorite!
%\usetheme{Boadilla} % Tretty neat, soft color.
%\usetheme{default}
\usetheme{Warsaw}
%\usetheme{Bergen} % This template has nagivation on the left
%\usetheme{Frankfurt} % Similar to the default with an extra region at the top.
%\usecolortheme{seahorse} % Simple and clean template
%\usetheme{Darmstadt} % not so good
% Uncomment the following line if you want page numbers and using Warsaw theme
% \setbeamertemplate{footline}[page number]
\setbeamercovered{transparent}
%\setbeamercovered{invisible}
% To remove the navigation symbols from the bottom of slides%
\setbeamertemplate{navigation symbols}{} 
}

\usepackage{ifpdf}
\usepackage{ifluatex}
\ifluatex
\usepackage{polyglossia}   
\setmainlanguage{polish}
\usepackage{amsmath}
\usepackage{mathtools}
\def\bm{\boldsymbol}
\usepackage{lualatex-math}
\else
\usepackage{graphicx}
\usepackage{polski} 
\usepackage[utf8x]{inputenc}
\usepackage{bm} 
\fi
\usepackage{ucs}
\usepackage{listings}
\usepackage{hyperref}
\usepackage{ragged2e}
%\apptocmd{\frame}{}{\justifying}{} % Allow optional arguments after frame.
%\apptocmd{\block}{}{\justifying}{} % Allow optional arguments after frame.
% For typesetting bold math (not \mathbold)

\usepackage{pgf,tikz}
\usetikzlibrary{graphdrawing}
\usetikzlibrary{graphs}
\usegdlibrary{trees}
\usetikzlibrary{intersections}
\usetikzlibrary{arrows,shapes,automata,positioning,calc}
\usepackage{pgfplots}
\usepackage{pgfplotstable}
%\usepackage[active,tightpage,pdftex]{preview}
%\PreviewEnvironment{tikzpicture}
\newcommand{\mycircle}[1]{\tikz{\filldraw[draw=#1,fill=#1] (0,0) circle [radius=0.3em];}}
\DeclareMathOperator{\sgn}{sign}
\usepackage{tkz-euclide}

%\logo{\includegraphics[height=0.6cm]{yourlogo.eps}}
%
\title[EARIN]{Data Mining Lectures - Support Vector Machines}
%
\author{Piotr Wąsiewicz}
\institute[ICS PW]
{
Institute of Computer Science\\
\medskip
{\emph{pwasiewi@elka.pw.edu.pl}}
}
\date{\today}
% \today will show current date. 
% Alternatively, you can specify a date.

\begin{document}
\lstset{language=R}

\begin{frame}
\titlepage
\end{frame}

\pgfmathsetseed{1138} % set the random seed
\pgfplotstableset{ % Define the equations for x and y
    create on use/x/.style={create col/expr={42+2*\pgfplotstablerow}},
    create on use/y/.style={create col/expr={(0.6*\thisrow{x}+130)+5*rand}}
}
% create a new table with 30 rows and columns x and y:
\pgfplotstablenew[columns={x,y}]{30}\loadedtable

\begin{frame}
\frametitle{Linear regression}
\begin{tikzpicture}
\begin{axis}[
xlabel=Weight (kg), % label x axis
ylabel=Height (cm), % label y axis
axis lines=left,    %set the position of the axes
xmin=40, xmax=105,  % set the min and max values of the x-axis
ymin=150, ymax=200, % set the min and max values of the y-axis
clip=false
]
\addplot [only marks] table {\loadedtable};
\addplot [no markers, thick, red] table [y={create col/linear regression={y=y}}] {\loadedtable} node [anchor=west] 
{$\pgfmathprintnumber[precision=2, fixed zerofill]{\pgfplotstableregressiona} \cdot \mathrm{Weight} + 
\pgfmathprintnumber[precision=1]{\pgfplotstableregressionb}$};
\end{axis}
\end{tikzpicture}
\begin{block}{}
\begin{itemize}
\item[\ ] Linear Regression $Height=g(Weight)$ finds a trend in data.
\end{itemize}
\end{block}
\end{frame}

\begin{frame}
\frametitle{Linear regression classifiers}
\begin{itemize}
\item Two-class classification - domain linear partition into the positive and negative regions by comparing against the threshold $g(a)=\langle \mathbf w \mathbf a \rangle = \langle w_{n+1}^1 a_{n+1}^1\rangle$, where $a_{n+1}=1$ for the intercept term $b=w_0=w_{n+1}$ from $g(a)=w_n^1a_n^1+w_0$
\[
h(a) = H(g(a)) = \left\{\begin{matrix} 1  & \text{if\ } g(a) \geq 0 \\ 0 & \text{otherwise} \end{matrix}\right.
\]
\item Class probability modeling - the logit representation $logit(p)=\ln{\frac{p}{1-p}}$, especially its inverse function called the logistic function
\[
P(1|a)=logit^{-1}(g(a))=\frac{e^{g(a)}}{e^{g(a)}+1}
\]
\end{itemize}
\end{frame}

\begin{frame}
\frametitle{Logistic function}
\begin{tikzpicture} 
\begin{axis}[ xlabel=$g(a)$, ylabel={$f(a)$}, 
xmin=-5, xmax=5, ymin=-0.15, ymax=1.15, legend pos=outer north east]
\addlegendimage{green, mark=x}
\addplot[color=black,mark=-] coordinates { (0,-0.15) (0,1.15) }; 
\addplot[color=green,mark=x] {e^x/(e^x+1)}; 
\addlegendentry{$\frac{e^{g(a)}}{e^{g(a)}+1}$}
\end{axis} 
\end{tikzpicture} 
\begin{block}{}
\begin{itemize}
\item The logistic function estimates probability (e.g. needed for ROC) of logical regression classification results.
\end{itemize}
\end{block}
\end{frame}

\begin{frame}
\frametitle{VC (Vapnik–Chervonenkis) dimension of hyperplanes in $\mathbb R^n$}
  \begin{itemize}
  \item For 4 points in 2D (a square or a matrix 2x2) and two classes:
    \begin{itemize}
    \item it is not always possible to separate them with lines. 
	\item[\ ]
	\begin{tabular}{cc}
	$c_1$ & $c_2$ \\
	$c_2$ & $c_1$ \\
	\end{tabular}
	$==>$
	\begin{tabular}{c|c}
	$c_1$ & $c_2$ \\
	$c_2$ & $c_1$ \\
	\end{tabular}
	$or$
	\begin{tabular}{cc}
	$c_1$ & $c_2$ \\
	\hline
	$c_2$ & $c_1$ \\
	\end{tabular}
    \end{itemize}
  \item $n+1$ points in  $\mathbb R^n$ can not be separated linearly
  \item The VC dimension of hyperplanes in $\mathbb R^n$ is $n + 1$
  \end{itemize}
\end{frame}


\begin{frame}
\frametitle{Support Vector Machines: basic linear examples}
  \begin{itemize}
  \item Linear separation for two classes
  \item Some training data $\{\mathbf a_i, c_i\}^i_m$, $\mathbf a_i \in \mathbb{R}^n$, and $c_i \in \{ -1, 1\}$
  \item Train to obtain the separation hyperplane:
    \begin{itemize}
    \item Minimize $d_+ + d_-$ to receive the shortest distances from the hyperplane 
    \begin{itemize}
    \item to closest positive point $d_+$
    \item to closest negative point $d_-$
    \end{itemize}
    \end{itemize}
  \item The goal is to find the separating hyperplane: $\mathbf w \mathbf a + b = 0$, where
    \begin{itemize}
    \item a vector $w$ is normal to the hyperplane
    \item $\displaystyle\frac{|b|}{\|\mathbf w\|}$ is the distance to origin $(0,0)$
    \item $\|\mathbf w\|$ the length of the vector $\mathbf w$
    \end{itemize}
  \item Designing a road between trees on the left and rocks on the right.
  \end{itemize}
\end{frame}

\begin{frame}
\frametitle{Support Vector Machines}
\begin{tikzpicture}[>=stealth']
  % draw axes
  \draw [<->,thick] (0,5) node (yaxis) [above] {$a_2$}
        |- (5,0) node (xaxis) [right] {$a_1$};
  % draw line
  \draw (0,-1) -- (5,4); % y=a-1
  \draw[dashed] (-1,0) -- (4,5); % y=a+1
  \draw[dashed] (2,-1) -- (6,3); % y=a-3
  % \draw labels
  \draw (3.5,3) node[rotate=45,font=\small] 
        {$\mathbf{w}\cdot \mathbf{a} + b = 0$};
  \draw (2.5,4) node[rotate=45,font=\small] 
        {$\mathbf{w}\cdot \mathbf{a} + b = 1$};
  \draw (4.5,2) node[rotate=45,font=\small] 
        {$\mathbf{w}\cdot \mathbf{a} + b = -1$};
  % draw distance
  \draw[dotted] (4,5) -- (6,3);
  \draw (5.25,4.25) node[rotate=-45] {$\frac{2}{\Vert \mathbf{w} \Vert}$};
  \draw[dotted] (0,0) -- (0.5,-0.5);
  \draw (0,-0.5) node[rotate=-45] {$\frac{b}{\Vert \mathbf{w} \Vert}$};
  \draw[->] (2,1) -- (1.5,1.5);
  \draw (1.85,1.35) node[rotate=-45] {$\mathbf{w}$};
  % draw negative dots
  \fill[red] (0.5,1.5) circle (3pt);
  \fill[red]   (1.5,2.5)   circle (3pt);
  \fill[black] (1,2.5)     circle (3pt);
  \fill[black] (0.75,2)    circle (3pt);
  \fill[black] (0.6,1.9)   circle (3pt);
  \fill[black] (0.77, 2.5) circle (3pt);
  \fill[black] (1.5,3)     circle (3pt);
  \fill[black] (1.3,3.3)   circle (3pt);
  \fill[black] (0.6,3.2)   circle (3pt);
  % draw positive dots
  \draw[red,thick] (4,1)     circle (3pt); 
  \draw[red,thick] (3.3,.3)  circle (3pt); 
  \draw[black]     (4.5,1.2) circle (3pt); 
  \draw[black]     (4.5,.5)  circle (3pt); 
  \draw[black]     (3.9,.7)  circle (3pt); 
  \draw[black]     (5,1)     circle (3pt); 
  \draw[black]     (3.5,.2)  circle (3pt); 
  \draw[black]     (4,.3)    circle (3pt); 
\end{tikzpicture}
\begin{block}{}
\begin{itemize}
\item $\max \frac{2}{\|w\|}$ - maximize the margin between parallel lines
\end{itemize}
\end{block}
\end{frame}

\begin{frame}
\frametitle{Support Vector Machines}
  \begin{itemize}
  \item $d_+$, $d_-$ the shortest distances from labeled points to hyperplane
  \item A margin $m = d_+ + d_-$
  \item The optimal separating hyperplane maximizes $m$ and minimizes the VC dimension
  \item For the separating hyperplane:
\setcounter{equation}{0}
    \begin{eqnarray}
      \mathbf w \mathbf a_i + b \geq +1,& c_i = +1\\
      \mathbf w \mathbf a_i + b \leq -1,& c_i = -1\\
      \equiv\\
      c_i(\mathbf w \mathbf a_i + b) - 1 \geq 0,& \forall i
    \end{eqnarray}
  \item For the closest points the equalities are satisfied, so:
    \begin{equation}
      d_+ + d_- = \frac{|1-b|}{\|w\|} + \frac{|-1-b|}{\|w\|} = \frac{2}{\|w\|}
    \end{equation}
  \end{itemize}
\end{frame}

\begin{frame}
\frametitle{Support Vector Machines: the linear example}
\setcounter{equation}{0}
\begin{minipage}[t]{.45\textwidth}
\begin{tikzpicture}[>=stealth']
  % draw axes
  \draw [<->,thick] (0,3.5) node (yaxis) [above] {$a_2$}
        |- (3.5,0) node (xaxis) [right] {$a_1$};
  % draw line
  \draw[dashed] (0,3) -- (2,3);   % class 1
  \draw[dashed] (2,0) -- (2,3);   % class 1
  \draw[dashed] (0,1) -- (1,1);   % class 2
  \draw[dashed] (1,0) -- (1,1);   % class 2
  \draw[red,thick] (0,2.75) -- (3.5,1);   % class 2
  \draw[red,dotted, ->] (2,3) -- (1,1); % margin
  % draw negative dots
  \fill[blue] (1,1) circle (2pt);  % class 2
  % draw positive dots
  \fill[black] (2,3) circle (2pt);% class 1
  % draw the optimum point between dots
  \fill[red] (1.5,2) circle (2pt);% class 1
  % \draw labels
  \draw (1,1) node[yshift=0.2cm,rotate=0,font=\tiny] {$(1,1)$};
  \draw (2,3) node[yshift=0.2cm,rotate=0,font=\tiny] {$(2,3)$};
  \draw (2,0) node[yshift=0.2cm,rotate=0,font=\tiny] {$(2,0)$};
  \draw (1.5,2) node[yshift=-0.8cm,rotate=63,font=\tiny] {$\mathbf w=(2,3)-(1,1)=(x,2x)$};
\end{tikzpicture}
\end{minipage}
\begin{minipage}[t]{.45\textwidth}
\begin{tikzpicture}[>=stealth']
  % draw axes
  \draw [<->,thick] (0,3.5) node (yaxis) [above] {$a_2$}
        |- (3.5,0) node (xaxis) [right] {$a_1$};
  % draw line
  \draw[dashed] (0,3) -- (2,3);   % class 1
  \draw[dashed] (2,0) -- (2,3);   % class 1
  \draw[dashed] (0,1) -- (1,1);   % class 2
  \draw[dashed] (1,0) -- (1,1);   % class 2
  \draw[red,thick] (0,2.75) -- (3.5,1);   % class 2
  \draw[red,dotted, ->] (2,3) -- (1,1); % margin
  % draw negative dots
  \fill[blue] (1,1) circle (2pt);  % class 2
  % draw positive dots
  \fill[black] (2,3) circle (2pt);% class 1
  % draw the optimum point between dots
  \fill[red] (1.5,2) circle (2pt);% class 1
  % \draw labels
  \draw (1,1) node[yshift=0.2cm,rotate=0,font=\tiny] {$(1,1)$};
  \draw (2,3) node[yshift=0.2cm,rotate=0,font=\tiny] {$(2,3)$};
  \draw (2,0) node[yshift=0.2cm,rotate=0,font=\tiny] {$(2,0)$};
  \draw (1.5,2) node[yshift=0.4cm,rotate=-27,font=\tiny] {$\mathbf g(a) = \frac{2}{5}a_1+\frac{4}{5}a_2-\frac{11}{4} - SVM!$};
  \draw (1.5,2) node[red,yshift=-0.4cm,rotate=-27,font=\tiny] {$\mathbf w = (\frac{2}{5},\frac{4}{5}) = support\ vectors!$};
\end{tikzpicture}
\end{minipage}
\begin{minipage}[t]{.45\textwidth}
\tiny
    \begin{eqnarray}
	  \mathbf w = (2,3) - (1,1) = (x, 2x) \\
      for\ point\ (2,3): \mathbf w \mathbf a + b = +1 \\
	   2x + 6x + b = 1 \\
	   b = 1 - 8x \\
      for\ point\ (1,1): \mathbf w \mathbf a + b = -1 \\
	   x + 2x + b = -1 \\
	   x + 2x + 1 - 8x = -1 \\
	   x = \frac{2}{5}\ b = -\frac{11}{5}
    \end{eqnarray}
\end{minipage}
\begin{minipage}[t]{.45\textwidth}
\tiny
    \begin{eqnarray}
	\underline{Support\ vectors}: w = (\frac{2}{5},\frac{4}{5}) \\
	\underline{SVM}: g(a) = \mathbf w \mathbf a + b \\
	\underline{SVM}: g(a) = \frac{2}{5}a_1+\frac{4}{5}a_2-\frac{11}{4} \\
	\forall{c(a)\in c_1}\  g(a) >= 1 \\
	\forall{c(a)\in c_2}\  g(a) <= -1 
    \end{eqnarray} 
\end{minipage}
\end{frame}

\begin{frame}
\frametitle{Support Vector Machines: Lagrangian}
  \begin{itemize}
  \item The constraints is more easy to handle.
  \item Using dual form SVM models predictions and calculations can be performed without using any attribute values other than inside dot products.
  \item Instead of calculating model parameters Lagrange multipliers are used.
  \item Prepared for the kernel trick to swap the dot product with the kernel special function.
  \end{itemize}
\end{frame}

\begin{frame}
\frametitle{Optimization problem with constraints}
\begin{itemize}
\item Minimize f(a) as a primal optimization problem
\begin{equation*}
\begin{aligned}
& \underset{a}{\min}
& & f(a) \\
& \text{with constraints:}
& & g_i(a) = 0, \; i = 1, \ldots, m.
\end{aligned}
\end{equation*}
\item Lagrangian is made of f(a) and constraints: $L(a,\lambda)=f(a)+\lambda g(a)$
\item Minimize $L(a,\lambda)$ in a domain $a$ in a dual optimization problem.
\item After minimization with derivatives of Lagrangian $L(a,\lambda)$ equal to 0, maximize $L(a,\lambda)$ in a domain $lambda$.
\begin{equation*}
\begin{aligned}
& \underset{a}{\min}\ \underset{\lambda}{\max} &  & f(a)+\lambda g(a)
\end{aligned}
\end{equation*}
\end{itemize}
\end{frame}

\begin{frame}
\scriptsize
\frametitle{Optimization primal and dual problem example}
\begin{tikzpicture} 
\begin{axis}[ xlabel=$g(a)$, ylabel={$f(a)$}, 
xmin=-2.5, xmax=4.0, ymin=-2.0, ymax=4.0, legend pos=outer north east]
\addlegendimage{red,mark=x}
\addlegendimage{blue,mark=.}
\addlegendimage{green, mark=square}
\addlegendimage{green,mark=triangle}
\addlegendimage{black, mark=o}
%\addlegendimage{only marks, mark=o}
%\addlegendimage{no markers,blue}
% http://pgfplots.sourceforge.net/gallery.html
% use TeX as calculator: 
\addplot [only marks,mark=*] coordinates { (0,2) };
\addplot[color=black,mark=-] coordinates { (0,-4) (0,4) }; 
\addplot[color=black,mark=-] {0}; 
\addplot[color=red,mark=x] {x^2-x+2}; 
\addlegendentry{$f(a)$}
\addplot[color=blue,mark=.] {x}; 
\addlegendentry{$g(a)$}
\addplot[color=green,mark=square] {x^2-0.5*x+2}; 
\addlegendentry{$f(a)+\frac{1}{2} g(a)$}
\addplot[color=green,mark=triangle] {x^2-3*x+2}; 
\addlegendentry{$f(a)+\lambda g(a)$}
\addplot[color=black,mark=o] {x^2+2}; 
\addlegendentry{$f(a)+1 g(a)$}
\end{axis} 
\end{tikzpicture} 
\begin{block}{}
\begin{itemize}
\item $\lambda = 1$ for Lagrangian $\max_{\lambda>0}\min_{a}f(a)+\lambda g(a)$ dual optimal solution.
\item Result is the same for primal and dual problem solution: (0,2)
\end{itemize}
\end{block}
\end{frame}

\begin{frame}
\small
\frametitle{Support Vector Machines: Lagrangian}
\begin{itemize}
\item
For $\{\mathbf{a}_i, c_i\}\in\mathbb{R}^N\times\{-1, 1\}$ a primal problem minimization:
\setcounter{equation}{0}
\begin{align}
  \text{min}_{\mathbf{w}, b}&\qquad\frac 12||\mathbf{w}||^2\nonumber\\
\text{Subject to}&\qquad
c_i\left(\left\langle{\mathbf{w},\mathbf{a}}\right\rangle +
  b\right)\geq 1.\label{eq:1}
\end{align}
\item
An unconstrained problem with the Lagrange multipliers for minimization $\min_{\mathbf{w}, b} L(\mathbf{w}, b, \lambda)$
\begin{align}
  L(\mathbf{w}, b, \lambda) = \frac 12||\mathbf{w}||^2-\sum_i\lambda_i\left(c_i(\left\langle{\mathbf{w},\mathbf{a}}\right\rangle + b)-1\right),
\end{align}
  \item Convex quadratic programming dual problem $\max_{\lambda>0} L_D$
    \begin{align}
      L_D = \displaystyle\sum^l_{i=1} \lambda_i - \frac{1}{2}\displaystyle\sum_{i,j=1}^l \lambda_i \lambda_j c_i c_j \langle\mathbf a_i \mathbf a_j\rangle\\
	  C \geq \lambda_i \geq 0\\
	  \sum_{i} \lambda_{i}c_{i} =0 
    \end{align}
\end{itemize}
\end{frame}

\pgfplotsset{compat=1.12}
\begin{frame}
\frametitle{The dot product surface $x^2+y^2$ cut by a plane $x+y+4$}
\begin{tikzpicture}[
        %declare function={f(\x,\y)=20*\x*exp(-\x^2-\y^2);},
        declare function={f(\x,\y)=(\x)^2+(\y)^2;},
		%declare function={g(\x,\y)=(\x+1)^2+(\y-1)^2;},
        declare function={g(\x,\y)=(\x+\y+4;},
		    ]
    \begin{axis}[   width=250pt,
                    height=250pt,
                    domain=-2:2,
                    view={-50}{20}
                ]
       	\addplot3 [surf, opacity=.3, samples=50] {f(x,y)};
       	\addplot3 [mesh,red, opacity=.1, samples=50] {g(x,y)};
		\addplot3 [samples=50,mark=none, contour gnuplot={labels=false,levels={5},handler/.style={only marks}}] {g(x,y)};
		\addplot3[black] table[x index=0, y index=1, z expr={f(x,y)}] {dm_svm_contourtmp0.table};
    \end{axis}
\end{tikzpicture}
\end{frame}

\begin{frame}
\frametitle{Support Vector Machines: Lagrangian}
  \begin{itemize}
  \item From dual problem
    \begin{align*}
  \max_{\lambda>0}\min_{\mathbf{w},b}\qquad L(\mathbf{w}, b, \lambda)
    \end{align*}
  \item
to a quadratic optimization problem 
    \begin{align*}
\max_{\lambda>0}\sum_{i} \lambda_{i}-\frac{1}{2} \sum_{i,j} \alpha^{t}K\alpha\\
K_{ij}=\langle\mathbf a_i \mathbf a_j\rangle\\
\alpha_{i}=\lambda_{i}c_{i} 
    \end{align*}
  \item Solution only depends on support vectors $\mathbf{w}_{\lambda>0}$
  \item All others have $\lambda_i=0$ and can be moved arbitrarily far from the decision hyperplane or removed
  \item Such the dual problem is solved with the help of a gradient descent, which is appropriate for dot products. 
  \item The kernel trick can be applied and $K$ can be replaced by an inner product $\phi(x_{i}).\phi(x_{j})$ in a higher dimensional space.
  \end{itemize}
\end{frame}

\begin{frame}
\frametitle{Support Vector Machines: space transformation}
  \begin{itemize}
  \item Take points from $\mathbb R^d$ to some space $\mathcal H$:
\setcounter{equation}{0}
    \begin{equation}
      \Phi : \mathbb R^d \rightarrow \mathcal H
    \end{equation}
  \item Choose kernel function $K$ such that
    \begin{equation}
      K(a_i, a_j) = \Phi(\mathbf a_i) \Phi(\mathbf a_j)
    \end{equation}
  \item Since in the Lagrangian formula we only have $a_i$ in dot products, we don't even need to know $\Phi$!
  \end{itemize}
\end{frame}

\begin{frame}\frametitle{Support Vector Machines: kernel}
  \begin{itemize}
  \item Replacing $\langle\mathbf a_i \mathbf a_j\rangle$ with  $K(\mathbf a_i, \mathbf a_j)$ everywhere do all the magic.
  \item Training is identical and takes almost similar time.
  \item Separation is still linear, but in a different space (infinite-dimensional!)
  \item Kernel examples:
  \begin{itemize}
  \item Gaussian Kernel
\setcounter{equation}{0}
    \begin{equation}
      K(\mathbf a_i, \mathbf a_j) = e^\frac{-\|\mathbf a_i - \mathbf a_j\|^2}{2\sigma^2}
    \end{equation}
  \item Polynomial
    \begin{equation}
      K(\mathbf a_i, \mathbf a_j) = (\gamma\langle\mathbf a_i \mathbf a_j\rangle + b)^p
    \end{equation}
  \item Based on neural net elements
    \begin{equation}
      K(\mathbf a_i, \mathbf a_j) = \tanh(\kappa\langle\mathbf a_i \mathbf a_j\rangle - \delta)^p
    \end{equation}
  \end{itemize}
  \end{itemize}
\end{frame}

\begin{frame}\frametitle{SVM For Multiple Classes}
  \begin{itemize}
  \item Build $n$ ``1-vs-all'' classifiers:
  \begin{itemize}
  \item It costs $n$ times the complexity of one classifier and the most confident answer should be chosen.
  \end{itemize}
  \item Build $\frac{n(n-1)}{2}$ ``1-vs-1'' classifiers:
  \begin{itemize}
    \item The instances are assigned by voting, so many classifiers have a small number of instances.
  \end{itemize}
  \item Large Margin DAG’s for Multiclass Classifications (Platt) means the Decision Directed Acyclic Graph (DDAG), which is used to combine many two­class classifiers into a multiclass classifier.
  \item Probabilities are calibrated by logistic regression on the SVM’s scores.
  \end{itemize}
\end{frame}

\begin{frame}\frametitle{Conclusions}
  \begin{itemize}
  %http://scikit-learn.org/stable/modules/svm.html#scores-probabilities
  \item Effective in cases where number of dimensions is greater than the number of samples.
  \item Support Vector Machines have different performance depending on the scaling of the data.
  \item Uses a subset of training points in the decision function (called support vectors), so it is also memory efficient.
  \item Complexity dependent on the number of support vectors (and also the kernel type) and is generally between $O(n^2)$ and $O(n^3)$ with $n$ the amount of training instances.
  \item Performance depends on choice of kernel and parameters.
  \item If the number of features is much greater than the number of samples, the method is likely to give poor performances.
  \end{itemize}
\end{frame}

\end{document}

\begin{frame}
\frametitle{}
\begin{block}{}
\begin{itemize}
\item
\end{itemize}
\end{block}
\end{frame}

\begin{frame}[fragile]
\scriptsize
%\includegraphics[width=11.8cm]{11gdiag\_01.jpg}
\frametitle{}
\begin{lstlisting}
\end{lstlisting}
\end{frame}

